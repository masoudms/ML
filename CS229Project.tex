\documentclass{beamer}
\usepackage{amsmath,amsfonts,amssymb}
\usepackage{mathtools}
\usepackage{array}
\usepackage{graphicx}
\usepackage{bbm}
\usetheme{Copenhagen}
% \usepackage{beamerthemesplit} // Activate for custom appearance

\title{Deep Q-Learning hyper-parameters optimization for portfolio management}
\author{CS 229 - Project}
\date{\today}

\setbeamercolor{block title}{bg=gray!30,fg=black}
\setbeamerfont{block title}{series=\bfseries}


\begin{document}

\frame{\titlepage}

\section[Outline]{}
\frame{
   \tableofcontents
}

\section{Introduction}
\frame{
  \frametitle{Introduction}
   
  \begin{itemize}
  	\item The proposed framework is to utlize machine learning methods to tune portfolio optimization parameters.
  	\item Bayesian Optimization (Gaussian Process) and Deep Q-Learning are considered for the purpose of this research.
	\item The tax-managed equity portfolio optimization will be studied as the underlying system.
   \end{itemize}
}
\section{The Tax-Managed portfolio optimization}
 \frame{
  \frametitle{The Tax-Managed portfolio optimization}
  	\begin{itemize}
		\item The tax-managed portfolio optimization aims to track the return of a benchmark while minimizing the tax impact.
		\begin{equation*}
			\begin{aligned}
			& \underset{w^p} {\text{minimize}}
			 & & TE^2 + \mathnormal{\sum_{i=1}^{M} ( {w_i}^0  - {w_i} ) TR_i (1 - \frac{c_i}{p_i})} \\
			 & \text{S.t:}
 			 & &\mathnormal{LB \leq  \mathbf{w_A} \leq UB}   \text{\hspace{3mm} Assets bound constraints}\\
			 &&&\mathnormal{{LB^f} \leq  \mathbf{w^{\prime}_A B_f} \leq {UB^f}}   \text{\hspace{3mm} Factor f exposure constraints}\\
			 &&& \mathnormal{\sum_{k=1}^{m} w_k = w_i}  \text{\hspace{3mm} sum of (m) lots weights to asset i weight}\\
			 &&& \mathnormal{\mathbf{w_p} \geq 0 }  \text{\hspace{3mm} long-only constraints}\\
			 &&& \mathnormal{\sum_{} q_i \leq N }  \text{\hspace{3mm} max positions (N) constraint}
			\end{aligned}
		\end{equation*}
			\end{itemize}
}

\frame{
  \frametitle{The Tax-Managed portfolio optimization - Cont'd}
  	\begin{itemize}
		\item The mixed integer quadratic optimization goal is to minimize tax cost in addition to the square of tracking error : \\
			\hspace{10mm} $ TE^2  = \mathbf{w^{\prime}_A B} \mathbf{\Omega}_F \mathbf{B^{\prime} w_A}  + \mathbf{w_A^{\prime} D w_A} $
		\item The continuous Decision variables:
			\begin{itemize}
				\item ${w_j}^p$ Portfolio weights (at asset level)
				\item $w_i$ lot $i$ weight
			\end{itemize}
		\item The Binary Decision variables:
			\begin{itemize}
				\item $q_i = 1$ if ${w_j}^p > 0$
			\end{itemize}
		\item Other Variables/Constants:
			\begin{itemize}
				\item ${w_i}^0$ initial lot i weight
				\item $w_i$ final lot i weight
				\item $TR_i$ Tax Rate (long term or short term)
				\item $c_i$ cost basis (asset price at purchase)
				\item $p_i$ current price (current asset price)
			\end{itemize}
		
	\end{itemize}
}

\frame{
  \frametitle{The Tax-Managed portfolio optimization - Cont'd}
  	\begin{itemize}
		\item Other Variables/Constants (cont'd):
			\begin{itemize}
				\item $\mathbf{w_A} $ Active weights (vector of ${w_j}^p - {w_j}^b$)
				\item $\mathbf{\Omega}_F$ Factors Covariance Matrix
				\item $\mathbf{B}$ Factors Loadings/Exposures Matrix
			\end{itemize}
		\item the factors covariance matrix and exposures can be estimated using Principle components analysis of historical stocks returns
		\item we also can use the quandl covariance matrix. In this case we won't need to use factor analysis altogether. 
		
	\end{itemize}
}


\section{Bayesian Optimization}
 \frame{
  \frametitle{Bayesian Optimization}
  	
  }
  
  \section{Deep Q-Learning}
 \frame{
  \frametitle{Deep Q-Learning}
  	
  }
  \section{References}
 \frame{
  \frametitle{References}
  	
  }
  
\end{document}